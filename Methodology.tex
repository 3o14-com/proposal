\chapter{METHODOLOGY}

The proposed system for a decentralized social media platform includes the following blocks, utilizing an Agile development model to ensure flexibility and responsiveness to user feedback throughout the development process:

\section{Requirement Analysis}
Identify user needs, including secure data sharing, privacy controls, and support for decentralized communication. Analyze existing federated platforms such as Mastodon to integrate best practices. Define key features like user profiles, posts, and federation. This phase will involve iterative discussions with stakeholders to refine requirements continuously.

\section{System Design}
Create an architectural blueprint for a federated platform using the ActivityPub protocol. Define data flow for secure communication between servers. Design a user-friendly interface (UI) through wireframes and mockups. Design reviews will be conducted in sprints to gather feedback and make necessary adjustments.

\section{Development}
Implement the system in two main parts, following Agile sprints to allow for incremental development and regular reassessment of priorities:
\begin{itemize}
    \item \textbf{Backend Development:} Build APIs for features like user authentication, content creation, and federation. Ensure secure communication using data encryption. Development will be broken down into manageable tasks, with regular stand-up meetings to track progress.
    \item \textbf{Frontend Development:} Develop a responsive interface integrating with backend APIs. The frontend will be iteratively improved based on user testing and feedback collected during sprint reviews.
\end{itemize}

\section{Integration with Decentralized Protocols}
Use ActivityPub to enable federated interaction between instances. Integration will be tested in each sprint to ensure compatibility and functionality.

\section{Testing}
Conduct unit tests for individual components such as authentication and post sharing. Perform integration tests to validate interactions between backend, frontend, and federation protocols. Stress test the platform to assess scalability. Testing will be an ongoing process, with feedback loops incorporated into each sprint to address issues promptly.

\section{Deployment}
Deploy the platform by hosting federated instances. Provide tools for users to create new instances or join existing ones. Deployment will occur in stages, allowing for user feedback and adjustments before full-scale launch.

\section{Performance Evaluation}
Regularly monitor the platform’s performance, focusing on scalability, security, and user satisfaction. Iterate based on feedback and improvements in decentralized protocols. Performance evaluations will be conducted at the end of each sprint to inform future development cycles and enhancements.
