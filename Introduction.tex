\pagenumbering{arabic}
\chapter{INTRODUCTION}
\section{Background}
Scientific communication plays a vital role in advancing research and knowledge sharing across academic communities. Traditional social media platforms while effective for general communication often lacks specialized features necessary for scientific discource. The emergence of decentralized technologies particularly the ActivityPub \cite{ActivityPub} Protocol and the Fediverse presents an oppurtunity to create a more suitable platform for academic communication. 
\section{Gap Identification}
Current platforms for scientific communication face several limitations:

\begin{itemize}
  \item Limited accessibility of scientific communication to the broader population beyond niche communities.
  \item Insufficient support for mathematical expressions and scientific notations.
  \item Lack of integration with academic citation systems.
  \item Reliance on a third party for the protection and moderation of user data.
\end{itemize}

\section{Motivation}
To create a social media platform that empowers researchers and academics to communicate their scientific work effectively. By bridging the gap between specialized communities and the general public, the platform aims to promote the understanding and appreciation of cutting-edge research across a wider audience. Along with the ability to run individual servers by individual people/institutions without losing the ability to communicate between each other.

\section{Objectives}
\begin{itemize}
  \item Develop a federated social media platform using the ActivityPub protocol with support for mathematical and scientific typesetting.
  \item Streamline the server setup process to enable technically literate individuals to host their own servers with minimal effort.
\end{itemize}
